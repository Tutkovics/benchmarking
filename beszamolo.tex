\documentclass{beamer}

%set up theme & used package
\usepackage[utf8]{inputenc}
\usepackage{textcomp}
\usepackage{graphicx}
\usepackage{caption}
\usepackage{verbatim}
\usepackage{wrapfig}
\graphicspath{ {images/} }

\usetheme[progressbar=frametitle,block=fill]{metropolis}
\setbeamertemplate{frame numbering}[fraction]
\usecolortheme{spruce}
\setbeamercolor{background canvas}{bg=white}
\useoutertheme{metropolis}
\useinnertheme{metropolis}
\usefonttheme{metropolis}

%Presentation infos
\title[Szóbeli beszámoló]{Felhő alapú alkalmazások teljesítményének kiértékelése és modellezése}
\subtitle{Konzulens: Dr. Rétvári Gábor}
\author{Tutkovics András}
\institute[BME]{Budapesti Műszaki és Gazdaságtudományi Egyetem}
\date{2020.01.08. 14:00}

\begin{document}
% Figure -> Ábra
\captionsetup[figure]{name={Ábra}}

% ===================================================================
% Címoldal
\begin{frame}
\titlepage
\end{frame}

% ===================================================================
% Miért volt értelme a munkának
\begin{frame}[t]{Motiváció}
Változó igények, technológiák
\begin{center}
\begin{tabular}{ c c }
 \uncover<1->{Fejlesztés & Üzemeltetés \\ [0.8ex] 
 \hline}
 \uncover<1->{Monolitikus alkalmazások} & \uncover<1->{Virtuális gépek} \\ 
 \uncover<1->{$\downarrow$} & \uncover<1->{$\downarrow$} \\
 \uncover<1->{Mikroszolgáltatások} & \uncover<1->{Konténerek} \\    
\end{tabular}
\end{center}
\uncover<2->{\vspace{5mm} \textbf{Cél:} költségek optimalizálása}
\begin{flushright}
\uncover<2->{(pl: emberi erőforrás (üzemeltető), használt cpu/memória)}
\end{flushright}
\end{frame}




% ===================================================================
% Féléves feladatom ismertetése
\begin{frame}[t]{Elvégzett feladatok}
\begin{itemize}
\item<1-> Kubernetes megismerése
\begin{itemize}
	\item[] \textbf{Irodalomkutatás}, ismeretek bővítése
	\item[] \textbf{Erőforrás limitálás}, ütemezés
	\item[] \textbf{Skálázási módszerek} támogatottsága, használata
\end{itemize}
%\item<1-> Vertikális és horizontális skálázás bemutatása
%\begin{itemize}
%	\item[] Támogatottsága és használata Kubernetes felett
%\end{itemize}
\item<1-> Mérési \textbf{környezet összeállítása}
%\begin{itemize}
%	\item[] Architektúra kialakítása, telepítése
%\end{itemize}
\item<1-> \textbf{Vezérlő program} összeállítása
\item<1-> \textbf{Eredmények kiértékelése}
\item<1-> \textbf{Modell keresése} a skálázás becslésére
\begin{itemize}
	\item[] Modell \textbf{implementálása, ábrázolás} 
\end{itemize}
\end{itemize}
\end{frame}


% ===================================================================
% Kubernetes környezet nagyon absztrakt bemutatása
\begin{frame}[t]{Kubernetes bemutatása}
\begin{itemize}
\item Konténer orkesztrációs platform \\ \footnotesize{(konténerek indítása, skálázása, monitorozása, forgalomelosztás...)} \normalsize
\item Absztrakciós szintek (objektumok) \\[1ex]
\item[] \centering{Konténer $>$ Pod $>$ Deployment $\leftarrow$ Service}
\item \raggedright Erőforrások limitálása (ütemezés)
\begin{figure}
	\centering
	\includegraphics[height=0.4\textwidth]{resource_limitation}
\end{figure}
\end{itemize}
\end{frame}


%\begin{frame}[t]{Skálázási módok}
%\begin{figure}
%	\centering
%	\includegraphics[width=0.8\textwidth]{scaling}
%	\footnote{http://abhijitkakade.com/2019/04/horizontal-vs-vertical-scaling-azure-autoscaling} 
	%\caption{Vertikális és horizontális skálázás}
%\end{figure}
%\end{frame}

% ===================================================================
% Architektúra bemutatása
\begin{frame}[t]{Környezet összeállítása}
\begin{figure}
	\centering
	\includegraphics[width=0.9\textwidth]{architecture}
	%\caption{Mérési környezet felépítése}
\end{figure}
\end{frame}


% ===================================================================
% Tesztelt alkalmazások
\begin{frame}[t]{Tesztelt alkalmazások}
\begin{enumerate}
\item Nginx
	\begin{itemize}
	\item Egyszerű használat $\rightarrow$ könnyebb tervezés/fejlesztés
	\item Statikus (\textit{Hello World}) oldal
	\end{itemize}
\item NodeJS - Prímszámoló
	\begin{itemize}
	\item Nagy számításigényű alkalmazás
	\item Változtatható nehézség (\textit{build} után is)
	\item Eratoszthenész algoritmusa
	\end{itemize}
\item Apache
	\begin{itemize}
	\item Nagy, statikus webalkalmazás
	\item $\sim$11 Megabájt szöveges tartalom
	\item Memóriafogyasztás figyeléséhez
	\end{itemize}
\item NodeJS - \textit{Hello World}
	\begin{itemize}
	\item Már meglévő \textit{image}
	\item Azonos alap, más programmal
	\end{itemize}
\end{enumerate}
\end{frame}

% ===================================================================
% Horizontális skálázás
\begin{frame}[t]{Horizontális skálázás}
\begin{columns}
	\column{0.6\textwidth}
	\begin{figure}
		\centering
		\includegraphics[width=1\linewidth]{horizontal} 
 		%\caption{Prím számláló NodeJS alkalmazás - Horizontális skálázás}	
	\end{figure}

	\column{0.4\textwidth}
	\begin{itemize}
		\item Azonos QPS mellett
		\item Több konténer $\rightarrow$ magasabb CPU 
		\item[]
		\item[] $QPS: 600$
		\item[]
	\end{itemize}

\begin{tabular}{ c c c }
 $\#pod$ & $_{(s)}^{CPU}$ & $_{(ms)}^{V\acute{a}lasz}$ \\ [0.8ex] 
 \hline
 2 & 0.32 & 3.366 \\ 
 4 & 0.4  & 1.786 \\    
\end{tabular}
\end{columns}
Általában a több egység több erőforrást használ, de van alkalmazás, ami kivételt jelent.
\end{frame}




% ===================================================================
% Vertikális skálázás
\begin{frame}[t]{Vertikális skálázás}

\begin{columns}
	\column{0.6\textwidth}
	\begin{figure}
		\centering
		\includegraphics[width=1\linewidth]{vertical} 
 		%\caption{Prím számláló NodeJS alkalmazás - Vertikális skálázás}	
	\end{figure}

	\column{0.4\textwidth}
	\begin{itemize}
		\item Azonos QPS mellett
		\item Azonos erőforrás felhasználás 
		\item Max QPS megnő
		\item[]	
		\item[] $\#pod: 4, QPS: 300	$
		\item[]
	\end{itemize}

\begin{tabular}{ c c }
 $_{(mCPU)}^{Limit}$ & $_{(ms)}^{V\acute{a}lasz}$  \\ [0.8ex] 
 \hline
 100 & 3.489  \\ 
 200 & 2.014  \\    
\end{tabular}
\end{columns}
Semelyik alkalmazás esetében nem jelent (jelentősen) több erőforrás igényt adott terhelés mellett. 
\end{frame}



% ===================================================================
% Matematikai modell
\begin{frame}[t]{Horizontális skálázási modell}
1 pod $\rightarrow$ általános:
\begin{equation}
f_{1}(q):q \in [0;Q] \rightarrow \tilde{f_{k}}(q)=kf_{1}\left ( \frac{q}{k} \right ):q \in [0;kQ] 
\end{equation}
\begin{itemize}
\centering
\item[] $\downarrow$ \\
\end{itemize}

$x$ pod $\rightarrow$ általános:
\begin{equation}
f_{x}(q):q \in [0;Q] \rightarrow \tilde{f_{k}}(q)=\frac{k}{x}f_{x}\left ( q\frac{x}{k} \right ):q \in [0;\frac{k}{x}Q] 
\end{equation}
\end{frame}

% ===================================================================
% Mennyire sikeres a modell
\begin{frame}[t]{Modell illesztése}
\begin{figure}
	\centering
	\includegraphics[width=0.9\textwidth]{predict}
	%\caption{Hello World - NodeJS alkalamazás miodell illesztés}
\end{figure}
\end{frame}


% ===================================================================
% Konklúzió
\begin{frame}[t]{Összegzés}
\begin{itemize}
\item Horizontális skálázás vagy vertikális skálázás?
\begin{center}

\begin{tabular}{ c c }
 \centering
 $Horizont\acute{a}lis$ & $Vertik\acute{a}lis$  \\ [0.8ex] 
 \hline
 K8s alapból támogatja & K8s nem támogatja alapból  \\ 
 Több erőforrást használ(hat) & Azonos mennyiségű erőforrás  \\ 
 Nő a rendszer megbízhatósága & Változatlan marad \\   
\end{tabular}
\end{center}
\item[]
\item Előnyök + hátrányok $\rightarrow$ tervezői döntés
\item[]
\item Modell 
\begin{itemize}
\item Koncepció jó, további kutatást igényel
\item Vertikális skálázásra is lehet keresni
\end{itemize}
\end{itemize}
\end{frame}


%\begin{frame}[t]{Következtetések és további munka}
%\begin{itemize}
%\item Horizontális skálázás
%	\begin{itemize}
%	\item Alkalmazásfüggő eredmények
%	\item Gyorsul a kiszolgálás
%	\item Kubernetes támogatja
%	\end{itemize}
%\item Vertikális skálázás
%	\begin{itemize}
%	\item Nem nő jelentősen a kiszolgálás
%	\item Kubernetes részben támogatja
%	\end{itemize}
%\item Tervezői döntés a megválasztás
%\item Modell pontossága
%	\begin{itemize}
	
	
%	\item Közelíti a pontos eredményt
%	\item Van egy nem azonosított tényező $\rightarrow$ lineáris változás az eredményben
%	\end{itemize}
%\end{itemize}
%\end{frame}



% ===================================================================
% Bíráló kérdései
\begin{frame}[t]{Bírálói kérdés (1/2)}
Mutassa be egy Ön által választott alkalmazásra a két stratégia különbségét vagy egyezőségét a CPU és memória használat valamint válaszidő paraméterekre.
\begin{columns}
\column{0.5\textwidth}
	\begin{figure}
	\centering
	\includegraphics[width=1\textwidth]{nginx_cpu}
	\end{figure}
	\small
	\centering
	1pod-200mCPU $\rightarrow$ 2pod-200mCPU
	0.702ms $\rightarrow$ 0.635ms
\column{0.5\textwidth}
	\begin{figure}
	\centering
	\includegraphics[width=1\textwidth]{nginx_memory}
	\end{figure}
	\small
	\centering
	2pod-100mCPU $\rightarrow$ 4pod-100mCPU
	2.830ms $\rightarrow$ 1.011ms
\end{columns}
\end{frame}


\begin{frame}[t]{Bírálói kérdés (2/2)}
Hogyan lehet az, hogy a 14. és 15. ábrán az elméleti maximum CPU használat felett is vannak mérési
eredmények?
\begin{figure}
	\centering
	\includegraphics[width=0.8\textwidth]{different}
	%\caption{Hello World - NodeJS alkalamazás miodell illesztés}
\end{figure}
\end{frame}


\end{document}

